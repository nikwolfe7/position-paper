\section{Introduction}
Effective tools to assist second language (L2) aquisition have been the subject of much research due to their wide practical applicability \cite{higgins1983computer} \cite{levy1997computer} \cite{hubbard2008call}. From easing cross-cultural and interpresonal communication in travel, business, international relations and other professional domains, the development of a truly automated and scientifically robust language learning platform has become something of a holy grail in the field of language technology. Part of the mystique of this problem stems from the fact that we still have only rudimentary models of the actual neurological process of human language learning and understanding, and as such there are many different and competing theories of practical L2 acquisition \cite{pedersenfar} \cite{mitchell2013second}. 

The preeminent approach taken by popular computer-assisted language learning (CALL) applications such as Rosetta Stone \cite{vesselinov2009measuring} and Duolingo \cite{vesselinov2012duolingo} \cite{von2013duolingo} favors a didactic, linear pedagogy in which language is learned as an abstraction where simple grammatical rules and core vocabulary are iteratively combined and expanded upon to teach more complex linguistic concepts over time. Speech training and assessment are something of an afterthought for these systems, the implied assumption of course being that literacy in a given language is more important than oral competence. In contrast to this is the more orally immersive, memory-reflex oriented structure of a Pimsleur-method guide \cite{pimsleur2013learn} \cite{pimsleur1966testing} \cite{pimsleur1971psychology} \cite{godwin2010emerging} or the collaborative community-oriented approach of Busuu \cite{pino2011busuu} \cite{ketyi2013using} or the promising but ill-fated LiveMocha \cite{jee2009livemocha} \cite{liaw2011review}. Applications such as Carnegie Speech's NativeAccent\textregistered \ \cite{eskenazi2007nativeaccenttm} and Babbel employ more advanced speech processing techniques to assess the ``nativeness'' of a learner's pronunciation and provide corrective feeback. A large amount of academic research has also been done in the field of computer-assisted pronunciation training (CAPT) and the potential use of speech recognition for language learning \cite{franco1997automatic} \cite{minematsu2004pronunciation} \cite{van2016evaluating} \cite{mccrocklin2016pronunciation} \cite{leepersonalized} \cite{cincarek2009automatic} \cite{kim1997automatic} \cite{wolfeapplause} including notable systems for teaching Mandarin Chinese \cite{chen2004automatic} \cite{huimproved}, Hindi \cite{patil2016detection}, and Arabic \cite{maqsood2016complete} among others \cite{cucchiarini1997automatic} \cite{bernstein1990automatic}. 

While an exhaustive evaluation of the pros and cons of each of these systems is beyond the scope of this paper, we can broadly categorize their approaches as being focused on the detection and classification of pronunciation errors with the contextual aim of helping a learner develop a ``native-sounding'' accent. Learner feedback in systems like NativeAccent\textregistered \ is geared towards emphasizing native articulatory correctness, whereas most systems simply attempt to classify common mispronunciation errors and offer numeric scale assessments, e.g. a score between 0 and 100. It is arguable that offering numeric feedback is the effective equivalent of asking a learner to self-assess, which is an error-prone process at best \cite{eskenazi2007nativeaccenttm}.

A poignant, though perhaps simplistic critique of typical approaches to CAPT is that so-called ``nativeness'' is both a culturally loaded and ill-defined concept. Many of the people in the world who communicate in English do not speak with a British or American accent, and furthermore, why should they? It is arguable that intelligibility and understandability are better and more useful benchmarks to define. There are of course situations in academic and professional environments where grammatical and articulatory correctness are important, but this is a high bar to set for most practical purposes. In day-to-day communication or travel situations where the only metrics of success are showing respect and being properly understood, it is typically the effort that counts more than the place of articulation. Furthermore, the most readily applicable and expedient (though widely overlooked) competencies in L2 acquisition are listening and speaking skills \cite{brown1996performance} \cite{renukadevirole} \cite{feyten1991power} \cite{nunan2002listening} \cite{ferris1996academic}. We therefore suggest that a learner's oral intelligibility by first-language (L1) speakers be the gold standard by which any practically-oriented CALL or CAPT system is evaluated. 