\section{Proposed System} 
With the previous discussion in mind, the general language-learning platform we propose has three main sub-systems:
\begin{enumerate}
\item An intelligent spaced repetition based mobile application focusing on oral recall, bootstrapped with data from existing language learning courses and applications
\item A crowd-based collaborative data acquisition service to assess pronunciation and intelligibility of learners' speech in order to gather constructive feedback, and,
\item A systematic means by which a learner can be guided to figure out a language for which only limited or insufficient learning materials are currently available
\end{enumerate}

We can consider each of these items in more detail in turn.

\subsection{Mobile Education App}
The rise of mobile as a platform for self-directed education dictates that any new language learning application be built for smart devices from the get-go. At the same time, given the known efficacy of immersion training for oral language learning, we must consider how to simulate this sort of teaching style on a smart device. Pimsleur guides are better poised than didactic regiments like Duolingo for immersion training because they are oriented towards practical scenarios and are almost entirely in audio format. (In fact, Paul Pimsleur developed one of his original compact language guides in Akan for the Peace Corps in 1971, and the first author both used this guide as a Peace Corps Volunteer in Ghana and extended its basic structure to build additional language guides for 13 West African languages \cite{wolfeapplause} \cite{wikipimsleur} \cite{wolfeclap}). Of course, the lack of intelligent corrective feedback in Pimsleur guides is an admittedly critical flaw. 

Both Duolingo and Pimsleur employ spaced repetition, though clearly intelligent spaced repetition based on learner performance is preferable to a simple audio recording. Duolingo, notably, also employs the crowd for translation tasks and attempts to make the interaction fun \cite{vesselinov2012duolingo} \cite{garcia2013learning}. 

\subsubsection{Bootstrapping Existing Language Resources}

One crucial advantage of a Pimsleur-style guide for an immersive language training experience is the choice of content itself. Pimselur lesson content is chosen specifically to match real-world scenarios and order is based more on practical utility than grammatical complexity, e.g. phrases like ``Good morning'' and ``I don't understand'' are taught before simple nouns like ``boy'' and ``cat.'' Determining the correct ordering of practical introductory phrases will likely change for different languages, however it is unlikely that the words ``boy'' or ``cat'' will ever occur before simple greetings in a real-world setting. The structure of a Pimsleur guide can thus be effectively bootstrapped to create a modifiable template for most introductory language lessons, as implemented by Wolfe et al. for the Celebrate Language Audio Project (CLAP) \cite{wolfeclap}. 

Additional resources to bootstrap into a practical language learning application include the guides created by the U.S. Foreign Service Institute, the Defense Language Institute and the Peace Corps, all of which are in the public domain and freely available online \cite{peacecorps} \cite{fsi} \cite{livelingua}. University resources such as the Indiana University Center for Language Technology (CeLT) and the UCLA Language Materials Project et al. are also freely available online resources for language technologists to use \cite{ucla} \cite{celt}. Using only the resources listed above, we were able to amass nearly 700 gigabytes of data for 157 languages, totalling 29,227 raw audio files.

 







\subsection{Crowd Data Acquisition and Speech Assessment}
\subsection{Guided Language Discovery}