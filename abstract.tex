\begin{abstract} %TODO ALL
\noindent Language learning applications are among the most popular ways in which people are employing smart devices for self-directed education. While well-established services such as Duolingo, Pimsleur, Rosetta Stone, Babbel, Busuu et al. compete for market share and teach according to varying theories of L2 acquisition, few if any of these applications can claim to offer automatically generated pronunciation and intelligibility feedback in a way that effectively prompts and assists a learner to improve their oral abilities in a given language. It is generally agreed that automatically generating effective pronunciation feedback is a hard problem requiring sophisticated machine learning and signal processing techniques as well as large amounts of annotated data. While recent advances in low-resource speech technology are noteworthy, we argue this is short-sighted and that the prevailing experience in modern machine learning is that there is no data like more data. We therefore propose a divergent approach in which we emphasize speaking skills and immersion-style L2 acquisition, and argue that a concerted crowdsourcing campaign to gather the data required to do this in increasingly automated ways is an integral long-term component of any successful language learning platform or application. 
  
\end{abstract}
%\noindent{\bf Index Terms}: Insert terms here
